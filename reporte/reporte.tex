%%%%%%%%%%%%%%%%%%%%%%%%%%%%%%%%%%%%%%%%%
% Cleese Assignment (For Students)
% LaTeX Template
% Version 2.0 (27/5/2018)
%
% This template originates from:
% http://www.LaTeXTemplates.com
%
% Author:
% Vel (vel@LaTeXTemplates.com)
%
% License:
% CC BY-NC-SA 3.0 (http://creativecommons.org/licenses/by-nc-sa/3.0/)
% 
%%%%%%%%%%%%%%%%%%%%%%%%%%%%%%%%%%%%%%%%%

%----------------------------------------------------------------------------------------
%	PACKAGES AND OTHER DOCUMENT CONFIGURATIONS
%----------------------------------------------------------------------------------------

\documentclass[11pt]{article}
\usepackage[dvipsnames]{xcolor}
%%%%%%%%%%%%%%%%%%%%%%%%%%%%%%%%%%%%%%%%%
% Cleese Assignment
% Structure Specification File
% Version 1.0 (27/5/2018)
%
% This template originates from:
% http://www.LaTeXTemplates.com
%
% Author:
% Vel (vel@LaTeXTemplates.com)
%
% License:
% CC BY-NC-SA 3.0 (http://creativecommons.org/licenses/by-nc-sa/3.0/)
% 
%%%%%%%%%%%%%%%%%%%%%%%%%%%%%%%%%%%%%%%%%

%----------------------------------------------------------------------------------------
%	PACKAGES AND OTHER DOCUMENT CONFIGURATIONS
%----------------------------------------------------------------------------------------

\usepackage{lastpage} % Required to determine the last page number for the footer
\usepackage[most]{tcolorbox}
\usepackage{graphicx} % Required to insert images

\setlength\parindent{0pt} % Removes all indentation from paragraphs

\usepackage[most]{tcolorbox} % Required for boxes that split across pages

\usepackage{booktabs} % Required for better horizontal rules in tables

\usepackage{listings} % Required for insertion of code

\usepackage[dvipsnames]{xcolor}

\definecolor{codegreen}{rgb}{0,0.6,0}
\definecolor{codegray}{rgb}{0.5,0.5,0.5}
\definecolor{codepurple}{rgb}{0.58,0,0.82}
\definecolor{backcolour}{rgb}{1,1,1}
\lstdefinestyle{mystyle}{
    backgroundcolor=\color{backcolour},   
    commentstyle=\color{codegreen},
    keywordstyle=\color{magenta},
    numberstyle=\tiny\color{codegray},
    stringstyle=\color{codepurple},
    basicstyle=\ttfamily\footnotesize,
    breakatwhitespace=false,         
    breaklines=true,                 
    captionpos=b,                    
    keepspaces=true,                 
    numbers=left,                    
    numbersep=5pt,                  
    showspaces=false,                
    showstringspaces=false,
    showtabs=false,                  
    tabsize=2
}
\renewcommand{\lstlistingname}{Código}% Listing -> Algorithm
\lstset{style=mystyle}

\usepackage{etoolbox} % Required for if statements

%----------------------------------------------------------------------------------------
%	MARGINS
%----------------------------------------------------------------------------------------

\usepackage{geometry} % Required for adjusting page dimensions and margins

\geometry{
	paper=a4paper, % Change to letterpaper for US letter
	top=3cm, % Top margin
	bottom=3cm, % Bottom margin
	left=2.5cm, % Left margin
	right=2.5cm, % Right margin
	headheight=14pt, % Header height
	footskip=1.4cm, % Space from the bottom margin to the baseline of the footer
	headsep=1.2cm, % Space from the top margin to the baseline of the header
	%showframe, % Uncomment to show how the type block is set on the page
}

%----------------------------------------------------------------------------------------
%	FONT
%----------------------------------------------------------------------------------------

\usepackage[utf8]{inputenc} % Required for inputting international characters
\usepackage[T1]{fontenc} % Output font encoding for international characters

\usepackage[sfdefault,light]{roboto} % Use the Roboto font

%----------------------------------------------------------------------------------------
%	HEADERS AND FOOTERS
%----------------------------------------------------------------------------------------

\usepackage{fancyhdr} % Required for customising headers and footers

\pagestyle{fancy} % Enable custom headers and footers

\lhead{\small\assignmentClass\ifdef{\assignmentClassInstructor}{\ (\assignmentTitle)}{}} % Left header; output the instructor in brackets if one was set
\chead{} % Centre header
\rhead{\small\ifdef{\assignmentAuthorName}{\assignmentAuthorName}{\ifdef{\assignmentDueDate}{\assignmentDueDate}{}}} % Right header; output the author name if one was set, otherwise the due date if that was set

\lfoot{} % Left footer
\cfoot{\small Página\ \thepage\ de\ \pageref{LastPage}} % Centre footer
\rfoot{} % Right footer

\renewcommand\headrulewidth{0.5pt} % Thickness of the header rule

%----------------------------------------------------------------------------------------
%	MODIFY SECTION STYLES
%----------------------------------------------------------------------------------------

\usepackage{titlesec} % Required for modifying sections

%------------------------------------------------
% Section

\titleformat
{\section} % Section type being modified
[block] % Shape type, can be: hang, block, display, runin, leftmargin, rightmargin, drop, wrap, frame
{\Large\bfseries} % Format of the whole section
{\assignmentQuestionName~\thesection} % Format of the section label
{6pt} % Space between the title and label
{} % Code before the label

\titlespacing{\section}{0pt}{0.5\baselineskip}{0.5\baselineskip} % Spacing around section titles, the order is: left, before and after

%------------------------------------------------
% Subsection

\titleformat
{\subsection} % Section type being modified
[block] % Shape type, can be: hang, block, display, runin, leftmargin, rightmargin, drop, wrap, frame
{\itshape} % Format of the whole section
{(\alph{subsection})} % Format of the section label
{4pt} % Space between the title and label
{} % Code before the label

\titlespacing{\subsection}{0pt}{0.5\baselineskip}{0.5\baselineskip} % Spacing around section titles, the order is: left, before and after

\renewcommand\thesubsection{(\alph{subsection})}

%----------------------------------------------------------------------------------------
%	CUSTOM QUESTION COMMANDS/ENVIRONMENTS
%----------------------------------------------------------------------------------------

% Environment to be used for each question in the assignment
\newenvironment{question}{
	\vspace{0.5\baselineskip} % Whitespace before the question
	\section{} % Blank section title (e.g. just Question 2)
	\lfoot{\small\itshape\assignmentQuestionName~\thesection~continúa en la siguiente página\ldots} % Set the left footer to state the question continues on the next page, this is reset to nothing if it doesn't (below)
}{
	\lfoot{} % Reset the left footer to nothing if the current question does not continue on the next page
}

%------------------------------------------------

% Environment for subquestions, takes 1 argument - the name of the section
\newenvironment{subquestion}[1]{
	\subsection{#1}
}{
}

%------------------------------------------------

% Command to print a question sentence
\newcommand{\questiontext}[1]{
	\textbf{#1}
	\vspace{0.5\baselineskip} % Whitespace afterwards
}

%------------------------------------------------

% Command to print a box that breaks across pages with the question answer
\newcommand{\answer}[1]{
	\begin{tcolorbox}[breakable, enhanced]
		#1
	\end{tcolorbox}
}

%------------------------------------------------

% Command to print a box that breaks across pages with the space for a student to answer
\newcommand{\answerbox}[1]{
	\begin{tcolorbox}[breakable, enhanced]
		\vphantom{L}\vspace{\numexpr #1-1\relax\baselineskip} % \vphantom{L} to provide a typesetting strut with a height for the line, \numexpr to subtract user input by 1 to make it 0-based as this command is
	\end{tcolorbox}
}

%------------------------------------------------

% Command to print an assignment section title to split an assignment into major parts
\newcommand{\assignmentSection}[1]{
	{
		\centering % Centre the section title
		\vspace{2\baselineskip} % Whitespace before the entire section title
		
		\rule{0.8\textwidth}{0.5pt} % Horizontal rule
		
		\vspace{0.75\baselineskip} % Whitespace before the section title
		{\LARGE \MakeUppercase{#1}} % Section title, forced to be uppercase
		
		\rule{0.8\textwidth}{0.5pt} % Horizontal rule
		
		\vspace{\baselineskip} % Whitespace after the entire section title
	}
}

%----------------------------------------------------------------------------------------
%	TITLE PAGE
%----------------------------------------------------------------------------------------

\author{\textbf{\assignmentAuthorName}} % Set the default title page author field
\date{} % Don't use the default title page date field

\title{
	\thispagestyle{empty} % Suppress headers and footers
	\vspace{0.2\textheight} % Whitespace before the title
	\textbf{\assignmentClass \\ \assignmentTitle}\\[-4pt]
	\ifdef{\assignmentDueDate}{{\small \assignmentDueDate}\\}{} % If a due date is supplied, output it
	\ifdef{\assignmentClassInstructor}{{\large \textit{\assignmentClassInstructor}}}{} % If an instructor is supplied, output it
	\vspace{0.32\textheight} % Whitespace before the author name
}
 % Include the file specifying the document structure and custom commands

%----------------------------------------------------------------------------------------
%	ASSIGNMENT INFORMATION
%----------------------------------------------------------------------------------------

% Required
\newcommand{\assignmentQuestionName}{Pregunta} % The word to be used as a prefix to question numbers; example alternatives: Problem, Exercise
\newcommand{\assignmentClass}{Análisis de Imágenes Digitales} % Course/class
\newcommand{\assignmentTitle}{Agosto 2023} % Assignment title or name
\newcommand{\assignmentAuthorName}{Luis Alberto Ballado Aradias} % Student name

% Optional (comment lines to remove)
\newcommand{\assignmentClassInstructor}{Dr. Wilfrido Gómez-Flores} % Intructor name/time/description
\newcommand{\assignmentDueDate}{CINVESTAV - UNIDAD TAMAULIPAS} % Due date

%----------------------------------------------------------------------------------------

\begin{document}

%----------------------------------------------------------------------------------------
%	TITLE PAGE
%----------------------------------------------------------------------------------------

\maketitle % Print the title page

\thispagestyle{empty} % Suppress headers and footers on the title page

\newpage

%----------------------------------------------------------------------------------------
%	QUESTION 1
%----------------------------------------------------------------------------------------

\begin{question}

%\questiontext{Dibuja las siguientes señales}

  \textbf{Información}
  
\begin{enumerate}
\item{$x[n] = u[n+3] + 0,5u[n-1]$}
  \begin{center}
    \includegraphics[width=0.5\columnwidth]{graph0.png} % Example image
  \end{center}
\item{$x[n] = -1^{n} u[-n-2]$}
  \begin{center}
    \includegraphics[width=0.5\columnwidth]{graph1.png} % Example image
  \end{center}
\item{$x[n] = \sum_{i=0}^{\infty} 4\delta[n-3k-1] $}
  \begin{center}
    \includegraphics[width=0.5\columnwidth]{graph2.png} % Example image
  \end{center}
\end{enumerate}

\end{question}

%----------------------------------------------------------------------------------------
%	QUESTION 2
%----------------------------------------------------------------------------------------

\newpage
\begin{question}

%\questiontext{Describa todas las características que sean evidentes de los siguientes sistemas}

%--------------------------------------------
\begin{enumerate}
\item{$y[n] = 3x[n-1] + 2x[n-2] + 0,75x[n+4] - 3y[n-1]$}

  \answer{
  \begin{itemize}
  \item Es un Sistema Lineal
  \item El valor de la salida depende de valores futuros de la entrada, el sistema \textbf{tiene memoria} 
  \item Debido a que la salida depende de valores futuros de la entrada el sistema \textbf{no es causal}
  \item \textbf{Sistema Inestable} por la retroalimentación
  \item Variante en el tiempo
  \item Ecuación en diferencia
  \item Respuesta al impulso \textbf{Infinita} por la dependencia a los valores futuros
  \item Cuarto orden (4 bloques de memoria)
  \end{itemize}
  }
\item{$y[n] = x[n] cos\left[ \frac{n}{2\pi}\right]$}
  \answer{
  \begin{itemize}
  \item Sistema No lineal, tiene una función periódica
  \item Invariante en el tiempo
  \item Los valores de salida n dependen solo de valores de entrada en el momento n, \textbf{sistema sin memoria}
  \item La salida no depende de valores futuros, el sistema \textbf{es causal}
  \item Respuesta al impulso \textbf{Infinita}
  \item Primer orden
  \end{itemize}
  }
  
\item{$y[n] = 2n^{2}x[n] + n \times x[n+1]$}

  \answer{
  \begin{itemize}
  \item \textbf{No} es un Sistema Lineal por el termino cuadratico
  \item El valor de la salida depende de valores futuros de la entrada, el sistema \textbf{tiene memoria} 
  \item Debido a que la salida depende de valores futuros de la entrada el sistema \textbf{no es causal}
  \item \textbf{Sistema Inestable}
  \item Variante en el tiempo
  \item Respuesta al impulso \textbf{Infinita}
  \end{itemize}
  }
  
\end{enumerate}

%--------------------------------------------

\end{question}
\newpage
%----------------------------------------------------------------------------------------
%	QUESTION 3
%----------------------------------------------------------------------------------------

\begin{question}

%\questiontext{Calcular la transformada Z de las 3 señales y los 3 sistemas previamente descritos.}

\begin{equation}\label{eq:uno}
  x[n]=u[n+3]+0,5u[n-1]
\end{equation}

\answer{
  \[X[z] = x[n]\cdot z^{-n}\]
  \[X[z] = \sum_{k=-3}^{-\infty} 1\cdot z^{-k} + 0,5 \sum_{k=1}^{\infty} 1 \cdot z^{-k}\]
  \[X[z] = \sum_{n=-3}^{-\infty} \left(\frac{1}{z}\right)^{n}+0,5 \sum_{n=1}^{\infty} \left(\frac{1}{z}^{n}\right)\]
  A partir de la serie geométrica:
  \[ \sum_{n=0}^{N} r^{n} = \frac{1-r^{N+1}}{1-r} \implies  \frac{\left(\frac{1}{z}\right)^{-3}-0}{1-\frac{1}{z}}+\frac{0,5\left(\frac{1}{z}\right)^{1}-0}{1-\frac{1}{z}}\]
  \[X[z]=\frac{z^3}{1-z^{-1}}+\frac{0.5\cdot z^{-1}}{1-z^{-1}}\]
  \[= \frac{z^3+0,5\cdot z^{-1}}{1-z^{-1}}\]
  expresando en positivos
  \[= \frac{z^3+0,5\cdot z^{-1}}{1-z^{-1}}\cdot\frac{z}{z}=\frac{z^4+0,5}{z-1}\]
}

\begin{equation}\label{eq:dos}
  x[n]=-1^n u[-n-2]
\end{equation}

\answer{
  \[X[z] = -1^n u[-n-2]\cdot z^{-n}\]
  \[X[z] = \sum_{n=-2}^{-\infty} -1^n\cdot z^{-n} = \sum_{n=-2}^{-\infty}-\left(\frac{1}{z}\right)^n\]
  A partir de la serie geométrica:
  \[X[z]=\frac{\left(-\frac{1}{z}\right)^{-2}-0}{1-\left(-\frac{1}{z}\right)}=\frac{\frac{1}{z^{-2}}}{1+z^{-1}}=\frac{z^2}{1+z^{-1}}\cdot \frac{z}{z}\]
  expresando en positivos
  \[X[z]= \frac{z^3}{z+1}\]
}

\newpage
\begin{equation}\label{eq:tres}
  x[n]=\sum_{k=0}^{\infty} 4\delta[n-3k-1]
\end{equation}

\answer{
  \[X[z]=4 \sum_{k=0}^{\infty} z^{-3k-1} = 4\sum_{k=0}^{\infty} \left(\frac{1}{z}\right)^{-3k-1}\]
  A partir de la serie geométrica:
  \[X[z]=4\left(\frac{\left(\frac{1}{z}\right)^{-(3\cdot0)-1}}{1-\frac{1}{z}}\right)=\frac{4z}{1-z^{-1}\cdot\frac{z}{z}}=\frac{4z^2}{z-1}\]
}

\begin{equation}\label{eq:cuatro}
  y[n]=3x[n-1]+2x[n-2]+0.75x[n+4]-3y[n-1]
\end{equation}

\answer{
  \[y[n]+3y[n-1]=3x[n-1]+2x[n-2]+0.75x[n+4]\]
  \[Y[z]+3Y[z]\cdot z^{-1} = 3X[z]\cdot z^{-1}+2X[z]\cdot z^{-2}+0.75X[z]\cdot z^4\]
  \[Y[z]\left(1+\frac{3}{z}\right)=X[z]\left(\frac{3}{z}+\frac{2}{z^{-2}}+0.75z^4\right)\]
  \[\frac{Y[z]}{X[z]}=\frac{3z^{-1}+2z^{-2}+0.75z^4}{1+3z^{-1}}\cdot \frac{z}{z}=\frac{3+2z+0.75z^5}{z+3}\]
}

\begin{equation}\label{eq:cinco}
  y[n]=x[n] cos\left[\frac{n}{2\pi}\right]
\end{equation}

\answer{
  \[Y[z]=X[z] z^0 \cdot \frac{z^2-z\cdot cos\left[\frac{n}{2\pi}\right]}{z^2-2z(cos\left[\frac{n}{2\pi}\right])+1}\]
  \[\frac{Y[z]}{X[z]}=\frac{z^2-z\cdot cos\left[\frac{n}{2\pi}\right]}{z^2-2z(cos\left[\frac{n}{2\pi}\right])+1}\]
}

\begin{equation}\label{eq:seis}
  y[n]=2n^2\cdot x[n] + n\cdot x[n+1]
\end{equation}

\answer{
  \[Y[z]=2n^2 X[z] z^{-0} + nX[z]z^1 = X[z](2n^2+zn)\]
  \[\frac{Y[z]}{X[z]}=2n^2+zn\]
}

\end{question}

%----------------------------------------------------------------------------------------

%\assignmentSection{Bonus Questions}

%----------------------------------------------------------------------------------------
%	QUESTION 4
%----------------------------------------------------------------------------------------
\newpage
\begin{question}

  \questiontext{Realice un programa, en cualquier lenguaje que prefiera, que ejecute las siguientes tareas.}
  
  \begin{itemize}
  \item{Recibe como entrada en texto plano la descripción de una señal y de un sistema, discretos ambos.}
  \item{Dibuja la señal y la respuesta al impulso del sistema.}
  \item{Ejecuta la convolución entre ambas entradas y dibujar la señal resultante.}
  \end{itemize}

\end{question}

  Correr el script:
  \begin{itemize}
  \item Tener python >= 3.10 instalado
  \item instalar numpy, matplotlib, dependiendo del manejador de paquetes puede que sea con la instrucción pip3.10 install nombredelpaquete
    \begin{lstlisting}[language=Bash, caption=instalar dependencias]
      $pip install numpy
      $pip install matplotlib
  \end{lstlisting}
  \begin{lstlisting}[language=Bash, caption=Correr el programa]
    $python3.10 ss_script.py datos.txt
  \end{lstlisting}

  \item donde el archivo datos esta conformado por las descripciones de entrada:
  \begin{lstlisting}[language=Bash, caption=Archivo con datos]
    x[n]=u[n+3]+0.5*u[n-1]
    x[n]=-1^n*u[-n-2]
    x[n]=sigma(0,10,4*delta[n-3*k-1])
    y[n]=3x[n-1]+2x[n-2]+0.75*x[n+4]-3y[n-1]
    y[n]=x[n]*cos[n/2*pi]
    y[n]=2*n**2*x[n]+n*x[n+1]
  \end{lstlisting}
  \end{itemize}

  \begin{tcolorbox}[width=\textwidth,colback={red},title={**NOTA**},outer arc=0mm,colupper=white]    
    El programa nadamas grafica las señales descritas en el archivo datos.txt y \textbf{no efectua la respuesta al impulso ni realiza la convolución}  
    %\includegraphics[scale=0.5]{frogimage.png}
  \end{tcolorbox}   


  
  \newpage
  %\lstinputlisting[
%	caption=Script, % Caption above the listing
%	label=lst:luftballons, % Label for referencing this listing
%	language=Python, % Use Perl functions/syntax highlighting
%	frame=single, % Frame around the code listing
%	showstringspaces=false, % Don't put marks in string spaces
%	numbers=left, % Line numbers on left
%	numberstyle=\tiny, % Line numbers styling
%	]{luftballons.pl}

    


\end{document}
